\documentclass[oneside]{article}

\title{Why study Complex analysis?}

%  ╔══════════════════════════════════════════════════════════╗
%  ║                         packages                         ║
%  ╚══════════════════════════════════════════════════════════╝

\usepackage{inputenc}         % Used to compile any UTF-8 Character (and supports Unicode)
\usepackage{extarrows}         % Used to compile any UTF-8 Character (and supports Unicode)
\usepackage{stackengine}
\usepackage{scalerel}
\usepackage{amsmath, amssymb} % Used to write better mathematical expressions (ex. \mathbb{R})
\usepackage{stackrel}         % for left-right arrows, staking symbol above and bellow
\usepackage{mathrsfs}
\usepackage{euscript}
\usepackage[font=itshape]{quoting}
\usepackage{mathtools}
  \usepackage{enumitem}
\usepackage{amsthm}           % Used to create theorem enviroments.
\usepackage{graphicx}         % Let's you use images in LaTeX; ex the \includegraphics command.
\usepackage{hhline}
%\usepackage{luatexja}
%\usepackage{fontspec}
\usepackage{dsfont}
\usepackage{wasysym}          % emoji's!
\usepackage{float}            % package with ``H'' for figures and tables, ex. Images, tables, etc
%\usepackage{subcaption}      % More options with sub-captions.
\usepackage{setspace}         % More flexibility on spacing in document.
\usepackage{hyperref}         % let's me put hyperlinks
%\usepackage{tikz}             % for commutative diagrams https://tex.stackexchange.com/questions/419221/how-to-do-the-pushout-with-universal-property
\usepackage{tikz-cd}          % for commutative diagrams https://tex.stackexchange.com/questions/419221/how-to-do-the-pushout-with-universal-property
%\usepackage{quiver}           % for better commutative diagrams
\usepackage{bookmark}         % creating heading shortcut bar on the left as in word
\usepackage{longtable}        % create tables that extend past one page
%\usepackage{siunitx}         % required for alignment
%\usepackage{multirow}        % for multiple rows
\usepackage{microtype}        % makes nice text. Fixes some under-/over- filled box problems.
%\usepackage{fancyvrb}        % Let's you write code (different style than listings).
\usepackage{tcolorbox}        % Makes nice boxes for thms, cor, prop, or anything else.
\tcbuselibrary{theorems}
%\usepackage{enumitem}        % More flexibility on enumeration (don't know much).
\usepackage{xcolor}           % Colour text.
\usepackage{wrapfig}          % To wrap text around figure
\usepackage{listings}         % Create nice colour code blocks (customized in preamble)
\usepackage{thmtools}         % Customize theorem environment (in preamble).
%\usepackage{marvosym}        % Provide ``basic'' symbols (hazard, religious, smiley face etc.)
\usepackage{array}            % \begin{table}{>{\em}c c} -> 1st column italic (bfseries for bold)
%\usepackage[english]{babel}  % If I want to blindtext in english.
\usepackage{blindtext}
\usepackage[a4paper, total={6in, 8in}]{geometry}
%\usepackage[symbol]{footmisc} % For daggers and asterisk for footnotes. 1: *, 2: dagger, etc.
\usepackage{fancyhdr}
\usepackage{titlesec}
%\usepackage{pst-plot}          %To make nice plots: https://tex.stackexchange.com/questions/47594/how-can-i-make-a-graph-of-a-function#47596
\usepackage[answerdelayed]{exercise}
\usepackage{etoolbox}          % for Dynkin diagrams
\usepackage{dynkin-diagrams}   % for Dynkin diagrams

% \usepackage{CJKutf8} % for some chinese

% to import figures via: https://github.com/gillescastel/inkscape-figures
\usepackage{import}
\usepackage{pdfpages}
\usepackage{transparent}

% This command is the ONLY command imported via the packages snippet, think of it as a tiny package written out
\newcommand{\incfig}[2][1]{%
    \def\svgwidth{#1\columnwidth}
    \import{./figures/}{#2.pdf_tex}
}

\pdfsuppresswarningpagegroup=1

%  ╔══════════════════════════════════════════════════════════╗
%  ║                       book format                        ║
%  ╚══════════════════════════════════════════════════════════╝

\pagestyle{fancy}
\setlength\headheight{20pt}
\fancyfoot[C,CO]{\textcolor{gray}{Nathanael Chwojko-Srawley}}
\fancyfoot[R,RO]{\thepage}{}

%chapter header format
\titleformat{\chapter}[display]{\bfseries\Huge\itshape}{\thechapter}{0.5ex}
{
    \rule{\textwidth}{1pt}
    \vspace{1ex}
    \centering
}
[
\vspace{-0.5ex}
\rule{\textwidth}{0.3pt}
]


%  ╔══════════════════════════════════════════════════════════╗
%  ║                        devnagari                         ║
%  ╚══════════════════════════════════════════════════════════╝

% For devnagari font for that one example
\newcommand\dn{\catcode`\~=12
           \fontspec[Script=Devanagari,Mapping=velthuis-sanskrit]{Nakula}}


%  ╔══════════════════════════════════════════════════════════╗
%  ║                     Custom Commands                      ║
%  ╚══════════════════════════════════════════════════════════╝

% mathbb
\newcommand{\A}{\mathbb{A}}
\newcommand{\C}{\mathbb{C}}
\newcommand{\F}{\mathbb{F}}
\newcommand{\N}{\mathbb{N}}
\newcommand{\Q}{\mathbb{Q}}
\newcommand{\R}{\mathbb{R}}
\newcommand{\V}{\mathbb{V}}
\newcommand{\Z}{\mathbb{Z}}
\renewcommand{\H}{\mathbb{H}}

% mathcal
\newcommand{\co}{\mathfrak{o}} %mathcal{o} looks like wreath product, so I changed it to mathfrak
\newcommand{\CA}{\mathcal{A}}
\newcommand{\CB}{\mathcal{B}}
\newcommand{\CC}{\mathcal{C}}
\newcommand{\CD}{\mathcal{D}}
\newcommand{\CF}{\mathcal{F}}
\newcommand{\CG}{\mathcal{G}}
\newcommand{\CH}{\mathcal{H}}
\newcommand{\CI}{\mathcal{I}}
\newcommand{\CL}{\mathcal{L}}
\newcommand{\CN}{\mathcal{N}}
\newcommand{\CO}{\mathcal{O}}
\newcommand{\CP}{\mathbb{C}\mathrm{P}}
\newcommand{\CQ}{\mathcal{Q}}
\newcommand{\CR}{\mathcal{R}}
\newcommand{\CK}{\mathcal{K}}
\newcommand{\CS}{\mathcal{S}}
\newcommand{\CT}{\mathcal{T}}
\newcommand{\CV}{\mathcal{V}}
\newcommand{\CZ}{\mathcal{Z}}

%mathscr
\newcommand{\ff}{\mathscr{F}}

%Euscript
\newcommand{\EF}{\EuScript{F}}
\newcommand{\EL}{\EuScript{L}}
\newcommand{\EO}{\EuScript{O}}
\newcommand{\EC}{\EuScript{C}}

%mathfrak (lowercase)
\newcommand{\fa}{\mathfrak{a}}
\newcommand{\fb}{\mathfrak{b}}
\newcommand{\fc}{\mathfrak{c}}
\newcommand{\fd}{\mathfrak{d}}
\newcommand{\fm}{\mathfrak{m}}
\newcommand{\fn}{\mathfrak{n}}
\newcommand{\fo}{\mathfrak{o}}
\newcommand{\fp}{\mathfrak{p}}
\newcommand{\fq}{\mathfrak{q}}

%mathfrak (upper case)
\newcommand{\FA}{\mathfrak{A}}
\newcommand{\FB}{\mathfrak{B}}
\newcommand{\FP}{\mathfrak{P}}
\newcommand{\FC}{\mathfrak{C}}
\newcommand{\FD}{\mathfrak{D}}
\newcommand{\FM}{\mathfrak{M}}
\newcommand{\FN}{\mathfrak{N}}
\newcommand{\FO}{\mathfrak{O}}

% operatorname
\newcommand{\ev}{\operatorname{ev}}
\newcommand{\tr}{\operatorname{tr}}
\newcommand{\nm}{\operatorname{Nm}}
\newcommand{\op}{\operatorname{op}}
\newcommand{\ad}{\operatorname{ad}}
\newcommand{\im}{\operatorname{im}}
\newcommand{\id}{\operatorname{id}}
\newcommand{\ch}{\operatorname{ch}}
\newcommand{\ab}{\operatorname{ab}}
\newcommand{\SL}{\operatorname{SL}}
\newcommand{\GL}{\operatorname{GL}}
\renewcommand{\Re}{\operatorname{Re}}
\renewcommand{\Im}{\operatorname{Im}}
\newcommand{\rad}{\operatorname{rad}}
\newcommand{\vol}{\operatorname{vol}}
\newcommand{\Gal}{\operatorname{Gal}}
\newcommand{\Rep}{\operatorname{Rep}}
\newcommand{\Aut}{\operatorname{Aut}}
\newcommand{\Sym}{\operatorname{Sym}}
\newcommand{\Mod}{\operatorname{Mod}}
\newcommand{\Grp}{\operatorname{Grp}}
\newcommand{\Hom}{\operatorname{Hom}}
\newcommand{\Emb}{\operatorname{Emb}}
\newcommand{\Ext}{\operatorname{Ext}}
\newcommand{\End}{\operatorname{End}}
\newcommand{\Inn}{\operatorname{Inn}}
\newcommand{\Ind}{\operatorname{Ind}}
\newcommand{\Out}{\operatorname{Out}}
\newcommand{\Jac}{\operatorname{Jac}}
\newcommand{\Nil}{\operatorname{Nil}}
\newcommand{\Tor}{\operatorname{Tor}}
\newcommand{\Ann}{\operatorname{Ann}}
\newcommand{\Syl}{\operatorname{Syl}}
\newcommand{\Res}{\operatorname{Res}}
\newcommand{\lcm}{\operatorname{lcm}}
\newcommand{\conj}{\operatorname{conj}}
\newcommand{\disc}{\operatorname{disc}}
\newcommand{\stab}{\operatorname{stab}}
\newcommand{\kdim}{\operatorname{kdim}}
\newcommand{\coker}{\operatorname{coker}}
\newcommand{\Obj}{\operatorname{Obj}}
\newcommand{\Mor}[1]{\operatorname{Mor}(\textbf{#1})}
\newcommand{\Frac}{\operatorname{Frac}}
\newcommand{\Frob}{\operatorname{Frob}}
\newcommand{\Spec}{\operatorname{Spec}}
\newcommand{\mSpec}{\operatorname{mSpec}}
\newcommand{\Char}{\operatorname{char}}
\newcommand{\fHom}[1]{\operatorname{Hom}(#1, \_\_)}
\newcommand{\cofHom}[1]{\operatorname{Hom}(\_\_, #1)}
\newcommand{\trdeg}{\operatorname{trdeg}}
\newcommand{\Span}{\operatorname{span}}

%shorthands
\renewcommand{\bf}[1]{\textbf{#1}}
\newcommand{\qn}{\quad\newline}
\newcommand{\oln}{\overline}
\renewcommand{\phi}{\varphi}
\newcommand{\placeholder}{\_\_\_\_\_}
\newcommand{\set}[2]{\left\{#1 \ \middle|\ #2\right\}}

% connectors
\renewcommand{\mid}{\big|}
\newcommand{\sse}{\subseteq}
\newcommand{\subg}{\leqslant}
\newcommand{\supg}{\geqslant}
\newcommand{\ssne}{\subsetneq}
\newcommand{\isosubg}{\lesssim}
\newcommand{\nsse}{\not\subseteq}
\newcommand{\nsubg}{\trianglelefteq}
\newcommand{\nsupg}{\trianglerighteq}
\newcommand{\csubg}{\blacktriangleleft}
%\newcommand{\subgne}{\lneqslant}

% limits
\newcommand{\Lim}{\varprojlim}
\newcommand{\colim}{\varinjlim}
\newcommand{\coLim}{\varinjlim}
\newcommand{\Dlim}{\lim\limits_{\longrightarrow}}
\newcommand{\coDlim}{\lim\limits_{\longleftarrow}}


%for saturated ideals in the localization section (I don't like these, I think I'll delete)
\newcommand{\LIm}{{}^{\iota} }
\newcommand{\LPre}{{}^{\pi} }

% Arrows
\newcommand{\rw}{\rightarrow}
\newcommand{\Rw}{\Rightarrow}
\newcommand{\lw}{\leftarrow}
\newcommand{\Lw}{\Leftarrow}
\newcommand{\lrw}{\leftrightarrow}
\newcommand{\LRw}{\Leftrightarrow}
\newcommand{\rrw}{\rightrightarrows}
\newcommand{\acton}{\curvearrowright}
\newcommand{\actson}{\curvearrowright}
\newcommand\mapsfrom{\mathrel{\reflectbox{\ensuremath{\mapsto}}}}

% dcups
\newcommand{\dcup}{\sqcup}
\newcommand{\Dcup}{\bigsqcup}

%a nice large circle
\newcommand{\tikzcircle}[2][red,fill=black]{\tikz[baseline=-0.5ex]\draw[#1,radius=#2] (0,0) circle ;}%
\makeatletter
\newcommand*\bigcdot{\mathpalette\bigcdot@{.5}}
\newcommand*\bigcdot@[2]{\mathbin{\vcenter{\hbox{\scalebox{#2}{$\m@th#1\bullet$}}}}}
\makeatother

%somehow not a default command
\def\rddots{\mathstrut^{.^{.^{.}}}}

%% new delimiter
\DeclarePairedDelimiter{\ceil}{\lceil}{\rceil}


%  ╔══════════════════════════════════════════════════════════╗
%  ║                        tcolorbox                         ║
%  ╚══════════════════════════════════════════════════════════╝

\tcbuselibrary{most}

%colors
\definecolor{dkgreen}{rgb}{0,0.6,0}
\definecolor{gray}{rgb}{0.5,0.5,0.5}
\definecolor{mauve}{rgb}{0.58,0,0.82}
\definecolor{lightyellow}{rgb}{1,1,0.6}
\definecolor{reddish}{HTML}{A01A3A}

%Wanna implement this, but not working: https://tex.stackexchange.com/questions/547979/problem-numbering-environment

% create tcolor boxes
\tcbuselibrary{theorems}
\newtcbtheorem[number within=section]{thm}{Theorem}%
{colback=green!5,colframe=green!35!black,fonttitle=\bfseries}{th}
\newtcbtheorem[number within=section]{lem}{Lemma}%
{colback=blue!5,colframe=black!35!black,fonttitle=\bfseries}{lm}
\newtcbtheorem[number within=section]{prop}{Proposition}%
{colback=white!5,colframe=white!35!black,fonttitle=\bfseries}{pr}
\newtcbtheorem[number within=section]{cor}{Corollary}%
{colback=blue!5,colframe=blue!35!black,fonttitle=\bfseries}{co}
\newtcbtheorem[number within=section]{axiom}{Axiom}%
{colback=black!5,colframe=yellow!35!black,fonttitle=\bfseries}{ax}
\newtcbtheorem[number within=section]{defn}{Definition}%
{colback=red!5,colframe=red!35!black,fonttitle=\bfseries}{df}


%insure the exercises are properly numbered
\counterwithin{Exercise}{section}
\counterwithin{Answer}{section}


%Custom enviroments
 \newenvironment{note}
 {\begin{tcolorbox}[enhanced, sharp corners, colback=white , borderline={0.2pt}{0pt}{black}]
   \begin{quote}\textbf{Note}:
   }{
   \end{quote}\end{tcolorbox}}

 \newenvironment{intuition}
 {\begin{quote}\textbf{Intuition}:
   }{
   \end{quote}}

\newenvironment{titledBox}[1]
 {\begin{tcolorbox}[enhanced, sharp corners, colback=white , borderline={0.2pt}{0pt}{black}]
   \begin{quote}\textbf{#1}
   }{
   \end{quote}\end{tcolorbox}}

\def\exampletext{Example} % If English
\colorlet{colexam}{red!55!black} % Global example color


\newtcbtheorem[auto counter, number within=section]{example}{Example}{%
    empty,% Empty previously set parameters
    title={\exampletext \thetcbcounter},% use \thetcbcounter to access the counter
    attach boxed title to top left,
    minipage boxed title,
    boxed title style={empty,size=minimal,toprule=0pt,top=4pt,left=3mm,overlay={}},
    coltitle=colexam,
    fonttitle=\bfseries,
    before=\par\medskip\noindent,
    parbox=false,
    boxsep=0pt,
    left=3mm,
    right=0mm,
    top=2pt,
    breakable,
    pad at break=0mm,
    before upper=\csname @totalleftmargin\endcsname0pt,
    overlay unbroken={\draw[colexam,line width=.5pt] ([xshift=-0pt]title.north west) -- ([xshift=-0pt]frame.south west); },
    overlay first={\draw[colexam,line width=.5pt] ([xshift=-0pt]title.north west) -- ([xshift=-0pt]frame.south west); },
    overlay middle={\draw[colexam,line width=.5pt] ([xshift=-0pt]frame.north west) -- ([xshift=-0pt]frame.south west); },
    overlay last={\draw[colexam,line width=.5pt] ([xshift=-0pt]frame.north west) -- ([xshift=-0pt]frame.south west); }%
}{ex}



\def\prooftext{Proof} % If English
\NewDocumentEnvironment{Proof}{ O{} }
{
  \colorlet{colexam}{black!55!black} % Global example color
  \newtcolorbox[number within=section]{testproofbox}{% \newtcolorbox[use counter=testexample]{testexamplebox}{%
    % Example Frame Start
    empty,% Empty previously set parameters
    title={\emph{\prooftext} \thetcbcounter: #1},% use \thetcbcounter to access the testexample counter text
    % Attaching a box requires an overlay
    attach boxed title to top left,
       % Ensures proper line breaking in longer titles
       minipage boxed title,
    % (boxed title style requires an overlay)
    boxed title style={empty,size=minimal,toprule=0pt,top=4pt,left=3mm,overlay={}},
    coltitle=colexam,fonttitle=\bfseries,
    before=\par\medskip\noindent,parbox=false,boxsep=0pt,left=3mm,right=0mm,top=2pt,breakable,pad at break=0mm,
       before upper=\csname @totalleftmargin\endcsname0pt, % Use instead of parbox=true. This ensures parskip is inherited by box.
    % Handles box when it exists on one page only
    overlay unbroken={\draw[colexam,line width=.5pt] ([xshift=-0pt]title.north west) -- ([xshift=-0pt]frame.south west); },
    % Handles multipage box: first page
    overlay first={\draw[colexam,line width=.5pt] ([xshift=-0pt]title.north west) -- ([xshift=-0pt]frame.south west); },
    % Handles multipage box: middle page
    overlay middle={\draw[colexam,line width=.5pt] ([xshift=-0pt]frame.north west) -- ([xshift=-0pt]frame.south west); },
    % Handles multipage box: last page
    overlay last={\draw[colexam,line width=.5pt] ([xshift=-0pt]frame.north west) -- ([xshift=-0pt]frame.south west); },%
    }
  \begin{testproofbox}
}
{\end{testproofbox}\endlist}

%  ╔══════════════════════════════════════════════════════════╗
%  ║                    for Dynkin diagrams                   ║
%  ╚══════════════════════════════════════════════════════════╝
\def\row#1/#2!{#1_{\IfStrEq{#2}{}{n}{#2}} & \dynkin{#1}{#2}\\}
\newcommand{\tble}[1]{
   \renewcommand*\do[1]{\row##1!}
   \[
      \begin{array}{ll}\docsvlist{#1}\end{array}
   \]
}

\setlength{\parindent}{0pt} % don't like indentation infront of paragraphs
\setlength{\parskip}{1ex}   %so that there is still space between paragarphs


%  ╔══════════════════════════════════════════════════════════╗
%  ║                       bibliography                       ║
%  ╚══════════════════════════════════════════════════════════╝
\usepackage{csquotes}
\usepackage{comment}
\usepackage[backend=biber,citestyle=alphabetic,bibstyle=authortitle]{biblatex}

\addbibresource{bibliography.bib}


\author{Nathanael Chwojko-Srawley}



%import options: all, bibliography, book-formatting, booksSetting, customEnvironments, dynkin, hw-formatting, language-setup, newcommands, packages, preamble, proofs, settings, tcolorbox, test.avg,
\begin{document}


``What is your intuition on complex analysis?'' It is a question I asked many times, often receiving answers
depending on the domain of mathematics the person seem to enjoy the most: ``It's about solving complicated
integrals'', ``It is a special case of harmonic analysis'', ``it is useful for analytic number theory'', ``it
is the study of Riemann surfaces''. None of these answer felt satisfactory to me: many answers felt like it
deferred the responsibility for its purpose to another field of mathematics. Though each of these has given me
some motivation for its study, these explanations did not feel like they gave a good reason for why complex
analysis produces such rigid, one may even sometimes say magical, results. This requirement that the
intuitions should explain the ``beauty'' of complex analysis has kept me wobbly on my footing for complex
analysis, never feeling I can justify calling such a field \emph{analysis} when compared to the flexible
results offered in real analysis.

I believe I came across an explanation that finally grounds me and unifies all previous explanations given to
me. This article will expand the following sentence
\begin{center}
  \addtolength{\leftskip}{2cm}  % Adjust left margin
\addtolength{\rightskip}{2cm} % Adjust right margin
  Complex analysis is the completion of polynomials with the tools of real analysis applied to it.
\end{center}

\begin{titledBox}{Level of Math Knowledge}
  This article shall try its best to be as accessible as possible. A lot of the intuition shall stem
  from an understanding of what is real analysis and the basic properties of polynomials. In fact, the reader
  who has already taken a course in complex analysis may find this article most useful to unify their
  framework on complex analysis.
\end{titledBox}

\section{Polynomials}

To understand complex analysis, we start with the initial objects of study: polynomials. They naturally arise
in many contexts in mathematics:
\begin{enumerate}
  \item (Algebra) Polynomials are the objects that hold \emph{relation information}: given some variables $x_1, ..., x_n$,
    by using $+$ and $\cdot$  we combine them to form polynomials which represent \emph{relations} in algebra. For those more
    familiar with some abstract algebra, any ring is the quotient of a free ring where the kernel consists of
    the polynomials representing all relationships on the generators. Thus, understanding the properties of
    polynomials is of keen interest to algebraist.
  \item (Analysis) Polynomials can approximate continuous functions on any (compact) neighbourhood to an
    arbitrary degree of precision (by the Stone–Weierstrass theorem). In particular, on compact intervals we can take
    a \emph{sequence} of polynomials that uniformally converges to the continuous function. Thus,
    understanding the properties of polynomials is of keen interest to analysists
   \item (Geometry) Polynomials can inherently create shapes from their zero sets and their
     graphs\footnote{which themselves are just zero sets of a polynomial}. As
     polynomials are determined by their coefficients, they are determined by only finitely many values.
     From this perspective, polynomials form some of the ``simplest'' shapes that can be studied. Thus,
     understanding the properties of polynomials is of keen interest to geometers.
 \end{enumerate}


Let us focus on the special case where there is only one variable. Any polynomial $p(z)$ is naturally studied
when over the complex numbers $\C$: every polynomial over $\C$ breaks down
into linear terms:
\[
  p(z) = c(z-a_1)(z-a_2)\cdots (z-a_n)
\]
Thus, the information about $p(z)$ is encoded in the multi-set $\{c, a_1, ..., a_n\}$\footnote{We may want to
recover this set by taking $p(z) = 0$, but there is a complication here where if $a_i = a_j$ for some $i,j$.
Then the zero set ``forgets'' this information; for this the notion of \emph{schemes} more naturally keeps
track of multiplicity information} where $c$ is some constant multiple. Thus, the complex numbers are the
natural domain to study polynomials.

It is thus productive to use all the tools of mathematics in understanding the properties of polynomials. We
may continue the study of polynomials over $\C$\footnote{Or more algebraically, over an algebraically closed
field $\oln{k}$ of characteristic $0$} in a purely algebraic direction and go towards results such as
\emph{galois theory} and \emph{number theory}. We may also treat them geometrically, leading to \emph{algebraic
geometry} and \emph{scheme theory}. From an analysis perspective, we may treat polynomials as a family of
functions and ask multiple analytic questions. This approach will require a bit more care due to the
``infinite'' nature of analysis as compared to the ``finite'' nature of polynomials.


\section{From Polynomials to Complex Analysis}

We wish to be able to study polynomials through the traditional analytic tools (limits, convergence, family of
functions, bounding, interpolation, etc). We may want to define a natural topology for polynomials and study
the analytic properties of this topology; this is called the \emph{Zariski Topology}\footnote{It is given by
taking the zero sets of polynomials as the collection of closed sets}. However, this approach
has a couple of blockers, namely:

\begin{itemize}
  \item The Zariski topology is too coarse for analysis. The euclidean topology is ``made'' for
    the ideas of analysis: limits have single points, it is a metric space\footnote{Most spaces of functions
    shall be metric spaces, though they may sometimes be limited to norms or Fr\'echet spaces}, properties such as
    compactness can be described using simple analytic properties such as limits or closed and bounded, and
    continuous functions have many flexible properties (consider results such as Urysohn's lemma, that any
    closed set is the zero-set of a continuous function, or a functions is continuous if and only if it
    preserves limits). These properties are \emph{not} present in the natural topology induced by polynomials,
    the Zariski topology: most open sets are \emph{dense}, limits usually converge to infinitely many points,
    compactness becomes a much weaker property instead the stronger notion of \emph{properness} needs to be
    used, and so forth.

  \item Polynomials are not \emph{complete}: The limit of polynomials can be an arbitrary continuous function
    on a compact interval. Even more simply, if we keep adding a term $a_nx^n$ to a polynomial $a_0 + a_1x
    + a_2x^2 + \cdots$, the result is \emph{not} a polynomial.
  \item Similarly to the previous point, there are ``not enough'' polynomials for some fundamental results
    expected in analysis: the subset of polynomials functions on the unit ball is not closed (there are not
    enough polynomials to converge to) and there are not enough polynomials to have limit when uniformally
    converging\footnote{In particular, uniformally converging on compact sets, a technicality due to unstable
    boundary behaviour which would be covered in a course in complex analysis}.
\end{itemize}

The solution to this is by taking the ``right completion'' of polynomials in such a way that polynomials can
be embedded in this completion and retain most of their properties. The completion ought to be as minimal
as possible to be ``as close'' to polynomials as possible, while being large enough so that we can work with
our usual real-analysis tools:

\begin{enumerate}
  \item We first allow \emph{infinite polynomials}, namely power series. For those who are comfortable with
    limits from category theory, we take the set of polynomials and add all the limits of the diagrams:
    \[
      \colim_{n \to \infty} p_n(x)
    \]
    where $p_n(x)$ is a polynomial of degree $n$ whose coefficients match the coefficients of the polynomials of
    degree $k< n$.
   \item The next condition comes from how much of our results depend on an open (or closed) subset of our
     space, especially when studying geometric properties. For this reason, instead of looking at functions
     that are power series, it would be better to focus on functions that are \emph{locally power series},
     that is there exists a neighbourhood $U$ of a point such that $f|_U$ is equal to a power series.

     Functions that are locally power series are called \emph{analytic}.
\end{enumerate}

These two conditions give enough functions for the collection to be closed under uniform convergence, namely
if $f_n$ is a sequence of analytic functions and  $f_n \to f$ uniformally (on compact sets), then $f$ shall
still be analytic by Weiestrass Convergence Theorem. This family of functions on an open subset
$U \sse \C$ is called the \emph{Holomorphic functions}, and we shall denote it as $\CH(U)$. The family
$\CH(U)$ forms a complete metric space\footnote{Even stronger it forms a Fr\'echet space, which ought to be
  thought as spaces that are not nice enough to define a single norm, but nice enough to replace the norm by a
sequence of compatible semi-norms}, giving us a natural space in which we may ask analytic questions!

The first thing we should do is to see which properties of polynomials extend to properties of holomorphic
functions. As it turns out, we have done a good job finding a proper embedding for our polynomials and they
share many similar properties:
\begin{itemize}
  \item Polynomials over $\C$ are certainly unbounded (they shall go to infinity). By Louisville's Theorem, the
    same applies to holomorphic functions
  \item If a holomorphic function $f$ is bounded by a polynomial in growth (i.e. $|f(z)| \le |p(z)|$), then
    $f$ is a polynomial. This mirrors how a polynomial cannot be bounded by another polynomial unless it is of
    lower order or equal to it.
  % \item Polynomials on $\C$ are a subset of  rational polynomials on $\mathbb{P}_\C^1$, and many results become
  %      simpler when working with rational polynomials. Similarly, Holomorphic
  %      functions are a subset of meromorphic on $\CP^1$ and mirror many of the same properties of rational
  %      polynomials.
  \item A polynomial is determined by its roots up to a constant multiple. By \emph{Weierstrass factorization
    theorem}, every holomorphic decomposes into an infinite product of the form:
    \[
      f(z) = z^me^{g(z)}\prod_k^{\infty} \left[\left(1 - \frac{z}{a_k}\right)e^{p(z)}\right]
    \]
    where $g(z)$ is a polynomial.
  \item Though a holomorphic functions can decomposed as above, it is not determined by its roots (for example, $e^z$ has no
    roots). The next best result is  \emph{Cauchy's integral formula} which allows us to ``extract''
    information from holomorphic functions about the value at points.
  % \item Using the above result, any entire \emph{meromorphic function} (a holomorphic function with singularities,
  %   the simplest such examples is $1/z$) must be the quotient of two entire holomorphic functions. In
  %   particular
    \item Polynomials form an integral domains and rational polynomials form a field. Similarly, the family of
      holomorphic functions forms an integral domain, while the family of meromorphic functions forms a field.
      From an analysis perspective this is \emph{very} surprising: if $f,g$
      are real differentiable function and $fg = 0$, then it is very much not the case that we can conclude
      that $f = 0$ or $g = 0$. This reflects how holomorphic functions do indeed ``naturally'' extend
      the polynomials and the algebraic structures they form!
  \item The polynomial automorphisms of $\C$ are linear, and the holomorphic automorphisms of $\C$ are also
    linear. Hence, extending to holomorphic functions does not add additional automorphisms of $\C$.
  \item The open subsets of the Zariski topology are usually \emph{dense} where its compliment is of
    codimension $\le 1$. In the case of $1$ dimension, only a finite set of points is outside any
    (nontrivial) open set. Similarly, most holomorphic functions have an \emph{analytic continuation} that
    extends the definition of a holomorphic function defined on $U \sse \C$ to all of $\C$ minus possibly a
    countable set of (non-clustered) points\footnote{a point is said to be a \emph{cluster point} if any
      open neighborhood around the point contains infinitely many points. It is non-clustered, or sparse,
    otherwise}!
\end{itemize}

These results should give a sense of how we have done a good job in finding the right analytic space to
embed polynomials!

We can ask if there is any results from complex analysis that comes back to the
study of polynomials. This is indeed the case in multiple ways, but let us highlight one (infamous) example: the
\emph{Riemann-zeta function} is the power series function:
\[
  \zeta(s) = \sum_k^\infty \frac{1}{s^k}
\]
Through it, we can find many important results about the polynomials that govern the properties of primes.
Unfortunately, going into details would require going over the connection between polynomials and primes,
a topic better left for another blog, (or to get the full details
see~\cite[chapter~3]{nathanaelchwojko-srawleyEverythingYouNeed} for my exposition on this topic). More
generally number theory, through the use of analytic number theory and Elliptic functions, has produced some
incredible results in the theory of polynomials from the theory of complex analysis.



\section{From Complex Analysis to Other Fields}

In a complex analysis class, holomorphic functions are not introduced as the completion of polynomials: we say
that $f: \C \to \C$ is \emph{complex differentiable} or \emph{holomorphic} at $z \in \C$ if the
derivative-limit exists, that is holomorphic function are locally complex-linear. It is then
proven that a function is holomorphic on open $U \sse \C$ if and only if it is locally a power series. This
approach is certainly pedagogically better as a lot of the intuition given by the above extensively leans on
knowledge of real analysis and relies only on calculus. An advantage of the pedagogical approach is the direct
connection to the geometric properties of holomorphic functions (as they allow us to define differential forms),
and the direct connection to the ``dynamics'' of holomorphic functions (they contribute to the theory of
harmonic functions). As holomorphic functions is the analytic setting of polynomials, we can ask how complex
analysis allows us to be a bridge between polynomials and geometry and harmonic analysis. This is a very deep
connection and cannot be comprehensively summarized, but for the readers enjoyment I've compiled some
interesting snippets that should demonstrate how powerful a tool complex analysis is.


\subsection{To Geometry}



The approach via differentiation naturally shows that such functions have a
natural \emph{geometric} structure. Namely, we may define \emph{holomorphic forms} $f(z)dz$. Then any first
course in complex analysis shows that $f$ is holomorphic on $U$ if and only if $f(z)dz$ is closed (that is, a
$1$-form $\omega$ on $\R^2$ is closed if and only if $\omega = f(z)dz$ for a holomorphic function $f$). Thus,
($1$-dimensional) holomorphic functions can be defined purely geometrically!

We may hope that the study of $1$-dimensional complex manifolds, historically called Riemann surfaces, gives information
about polynomials. In fact, an absolutely incredible connection happens. When working with
\emph{compact sets} we usually expect a level of finiteness with what we are working with, be it boundedness,
growth, extreme values, and so forth. In the case of compact compact Riemann surfaces, the functions and
characterization of them reduces to the study of polynomials! In particular:

\begin{center}
    \addtolength{\leftskip}{2cm}  % Adjust left margin
    \addtolength{\rightskip}{2cm} % Adjust right margin
  Every compact Riemann surface is an algebraic curve, and two compact Riemann surfaces are (real)-diffeomorphic if
  they have the same number of holes (their genus is equal)
\end{center}

These results have major consequences in both algebra in geometry! For example, though not immediately
obvious, the study of equations of the form
\[
  y^2 = x^3 + ax + b
\]
is in fact the study of Compact Riemann surfaces of genus $1$, i.e. the study of tori! Such connections are
heavily exploited in the modern theory of algebraic geometry and would be further explored in a class on
algebraic curves or complex geometry.

% Let us tie this with how complex analysis is usually

\subsection{To Harmonic Analysis}

One of the first results in the theory of holomorphic functions is that a holomorphic function $f$ satisfies
the \emph{Cauchy-Riemann equations}
\[
  \frac{\partial f}{\partial x} = - i\frac{\partial f}{\partial y}
\]
In particular, if $f = u+iv$ is the decomposition of $f$ into to real functions, $u$ is a \emph{harmonic
function}:
\[
  \frac{\partial^2 u}{\partial x^2} + \frac{\partial^2 u}{\partial y^2} = \Delta u = 0
\]
where $\Delta$ is the Laplace operator. A classical exercise is to show that every harmonic function is the
real part of a holomorphic function. Thus, holomorphic functions have the \emph{dynamics} of harmonic PDEs,
that is, given a point $z \in U \sse \C$, the properties of the paths as this point travels in the field given
by $f$ will behave like those functions satisfying the Laplace operator. Hence, complex functions have the
\emph{harmonic dynamics}! A couple of striking consequences are shown in every first course in
complex analysis:

\begin{itemize}
  \item Every harmonic function is smooth, hence every holomorphic functions is smooth. Namely, it suffices to
    show it is once complex-differentiable to show it is infinitely complex-differentiable
  \item It satisfies the mean value property: The value at any point $f(z)$ is equal to the average around
    the point:
    \[
      f(a) = \int_{|z-a|=r} f(\gamma)d\gamma
    \]
  \item From this, we can also deduce that if $K \sse \C$ is closed and $U$ is holomorphic on $\int K$, it is
    harmonic on $K$ and it achieves it maximum/minimum on the boundary $\partial K$!
\end{itemize}

Using these results, we can ``see'' why $z^2+1$ must have a roots in $\C$. Taking a circle of radius
$1$, we see that the functions ``rises'' when going in the $x$-direction up to $2$. By the Cauchy-Riemann
equations and the mean value property, it must similarly ``dip'' in the $y$-direction\footnote{This intuition
  is a crude intuition and ultimately is relies on the rigidity of $z^2+1$, but it should still be an
enlightening example}; we can see how it would have roots at $\pm i$; see
\href{https://www.youtube.com/watch?v=T647CGsuOVU}{this video} for an excellent visualization of this
phenomenon!


\section{Conclusion}

Complex analysis touches so many fields. Depending on the readers interests, different aspects of complex
analysis will resonate to them. What is ultimately satisfying to me is that the ``weirdness'' of holomorphic
functions and their rather extreme rigidity as compared to real smooth functions can be put at the feet of the
rigidity of polynomials and the closeness between polynomials and holomorphic functions. This article
scratches the surface of this connection, for example no time has been spent elucidating the connection
between elliptic curves and harmonic functions\footnote{Fourier analysis makes some very interesting
appearances!}, and almost no time has been spent elucidating the bridges
algebra and geometry its connection to complex analysis\footnote{The connection to Riemann surfaces shows a
  small part of this connection, showing how a crude invariant of curves is their genus number, however the
Riemann-Roch theorem had no time to make an appearance!}. The reader interested in seeing the details of these
ideas expanded upon can checkout the multitude of textbook on complex analysis, or you can
checkout~\cite{NateComplexAnalysis} to see my exposition on the topics.



\printbibliography





\end{document}
